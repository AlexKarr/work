\documentclass{article}
%\usepackage[T2A]{fontenc}
\usepackage[utf8x]{inputenc}
\usepackage[russian]{babel}
\usepackage{amssymb,amsmath}
\begin{document}
\flushleft
\itshape
\slshape
\parindent=0.4cm
\textbf{\chaptername \hspace{0.1cm} {1}}\par
\Large
\upshape
\textbf{Спиноры в конечно мерных евклидовых пространствах.}\par
\vspace{1.5cm}
\normalsize
\S{1.}
\hspace{0cm}
\textbf{Алгебра $\gamma$-матриц.}\par
\vspace{0.3cm}

{Рассмотрим матричное уравнение}
\begin{center}
{$\gamma_i\gamma_j + \gamma_j\gamma_i=2\delta_{ij}$\itshape{I}\upshape \hspace{2cm} (1.1)}
\end{center}
в котором -квадратные, в общем случае комплексные,матрицы порядка $2^{\nu}$, $\nu$-целое положительное число; индексы i,j определяющие номера матриц $\gamma$, принимают все целочисленные значения от 1 до 2; \textsl{I} -- единичная матрица порядка $2^{\nu}$; $\delta_{ij}$ -- символы Кронекера:\par
\begin{center}
$\delta_{ij}$=1, если i=j,\par
$\delta_{ij}$=0, если i$\ne$j.     (1.2)\par
\end{center}
{Введем матрицы $\gamma_{i_1,i_2,\dots, i_k}$ определяемые при помощи матриц $\gamma_i$, удовлетворяющих уравнению     (1.1):}\par
\begin{center}
$\gamma_{i_1,i_2,\dots, i_k}$ = $\gamma_{[i}\gamma_{i_2}$ \dots $\gamma_{{i_k}]}$   (1.3)\par
\end{center}
{По индексам $i_1,i_2,\dots i_k$, помещенным в квадратные скобки в формуле (1.3), производиться альтернирование} {(по индексам k|)}.\par
Предполагая, что матрицы $\gamma_i$, удовлетворяющие уравнению (1.1), существуют, установим некоторые общие \par свойства матриц (1.3), не зависящие от конкретного вида ${\gamma_i}^1$).\par
1. Для каждой матрицы $\gamma_{i_1,i_2,\dots, i_k}$ существует такая матрица $\gamma_{j_1,j_2,\dots, j_m}$, что выполняется равенство\par
\begin{center}
$\gamma_{i_1,i_2,\dots, i_k}$$\gamma_{j_1,j_2,\dots, j_m}$=-$\gamma_{j_1,j_2,\dots, j_m}$$\gamma_{i_1,i_2,\dots, i_k}$ (1.4)\par
\end{center}
Это свойство может быть доказано непосредственным указанием для каждой матрицы $\gamma_{i_1,i_2,\dots, i_k}$ соответствующих антикоммутирующих с ней матриц. Если число индексов $i_1,i_2,\dots i_k$ четное k=2m, то с $\gamma_{i_1,i_2,\dots, i_k}$ антикоммутирует, например, любая матрица $\gamma_j$ при j, совпадающим с с одним из индексов $i_1,i_2,\dots i_k$. Если число индексов $i_1,i_2,\dots i_k$ нечетно, k=2m+1, то с $\gamma_{i_1,i_2,\dots, i_k}$ антикоммутирует любая матрица $\gamma_j$ при j, не совпадающим ни с одним из индексов $i_1,i_2,\dots i_k$.\par
2. Квадрат матриц $\gamma_{i_1,i_2,\dots, i_k}$ пропорционален единичной\par
матрице:
\begin{center}
$(\gamma_{i_1,i_2,\dots, i_k})^2$ = $(-1)^{{\frac{1}{2}}k(k-1)}$ \textsl{I}    (1.5)\par
\end{center}
{Равенство (1.5) следует непосредственно из уравнения (1.1) и определения (1.3).\par
3. Все матрицы $\gamma_{i_1,i_2,\dots, i_k}$ при k=1,2,\dots,$2\nu$ имеют след, равный нулю. Действительно, имея в виду, что для любых квадратных матриц \textsl{A,B,C} выполняется тождество\par
\begin{center}
tr(\itshape{ABC}\upshape)=tr(\itshape{CAB}\upshape),   (1.6)\par
\end{center}
найдем,учитывая равенство (1.5),\par
tr$\gamma_{i_1,\dots, i_k}=(-1)^{{\frac{1}{2}}m(m-1)}tr(\gamma_{i_1,\dots, i_k}\gamma_{j_1,\dots, j_m\gamma_{j_1,\dots, j_m}})$=\par
\begin{center}
$=(-1)^{{\frac{1}{2}}m(m-1)}tr(\gamma_{j_1,\dots, j_m}\gamma_{i_1,\dots, i_k\gamma_{j_1,\dots, j_m}})$.   (1.7)\par
\end{center}
Выбирая в (1.7) в качестве матрицы $\gamma_{j_1,\dots, j_m}$ матрицу, удовлетворяющую соотношению (1.4), продолжим равенство (1.7)\par
tr$\gamma_{i_1,\dots, i_k}=-(-1)^{{\frac{1}{2}}m(m-1)}tr(\gamma_{j_1,\dots,j_m}\gamma_{j_1,\dots, j_m\gamma_{i_1,\dots, i_k}}$)=\par
\begin{center}
=-tr$\gamma_{i_1,\dots, i_k}$   (1.8)\par
\end{center}
Из последнего равенства следует
\begin{center}
tr$\gamma_{i_1,i_2,\dots, i_k}$=0     (1.9)\par
\end{center}
4. След произведения двух матриц $\gamma_{i_1,i_2,\dots, i_k},\gamma_{j_1,j_2,\dots,j_k}$ отличен от нуля только в том случае, если эти матрицы одинаковы (или отличаются знаком за счет перестановки индексов при k=m).\par
В самом деле, если эти матрицы $\gamma_{i_1,i_2,\dots, i_k},\gamma_{j_1,j_2,\dots,j_m}$ различны, то их произведение в силу уравнения (1.1) сводится к некоторой матрице $\gamma_{q_1,q_2,\dots,q_n}$, след которой согласно (1.9), равен нулю. Если матрицы $\gamma_{i_1,i_2,\dots, i_k},\gamma_{j_1,j_2,\dots,j_m}$ одинаковы, то, согласно уравнению (1.5), их произведение пропорционально единичной матрице и след отличен от нуля. Сказанное можно записать в виде равенств:\par
tr($\gamma_{i_1,i_2,\dots, i_k}\gamma_{j_1,j_2,\dots,j_m}$)=0, если k$\ne$m,\par
tr($\gamma_{i_1,i_2,\dots, i_k}\gamma_{j_1,j_2,\dots,j_m})=-(-1)^{{\frac{1}{2}}k(k-1)}k!2^\nu\delta^{j_1}_{[{i_1}}\delta^{j_2}_{i_2}\dots\delta^{j_k}_{i_k]}$ (1.10)\par
Множитель k! в правой части второй формулы (1.10) связан с тем, что по индексам $i_1,i_2,\dots,i_k$ в этой формуле производится альтернирование.\par
5. Система матриц\par
\begin{center}
I,$\gamma_i,\gamma_{i_1 i_2},\dots,\gamma_{i_1 i_2\dots i_{2\nu}}$\hspace{0.2cm}($i_1<i_2<\dots<i_2\nu$)\hspace{0.2cm} (1.11)\par
\end{center}
линейно независима.\par
Действительно, рассмотрим уравнение
\begin{center}
$\alpha I+\sum_{k=1}^{2\nu}\alpha^{i_1 i_2\dots i_k}\gamma_{i_1 i_2\dots i_k}$=0,    (1.12)\par
\end{center}
в котором коэффициенты $\alpha^{i_1 i_2\dots i_k}$ антисимметричны по всем индексам: $\alpha^{i_1 i_2\dots i_k}=\alpha^{[i_1 i_2\dots i_k]}$. В уравнении (1.12) и во всем тексте далее (за исключением особых оговорок) используется правило суммирования, согласно которому по двум повторяющимся индексам производится суммирование по всем значениям, принимаемым этими индексами\par
Беря след от уравнения (1.12), найдем, учитывая (1.9), что коэффициент $\alpha$ равен нулю, $\alpha$=0. Беря след от уравнения (1.12), предварительно умноженного на матрицу $\gamma_{j_1 j_2 \dots j_m}$, с учетом равенств (1.9), (1.10) получим $\alpha^{j_1,j_2\dots j_m}$=0. Таким образом. если выполняется уравнение (1.12), то все коэффициенты $\alpha,\alpha^{i_1,i_2\dots i_k}$ в этом уравнении равны нулю, что и доказывает линейную независимость системы матриц (1.11).\par
Очевидно, что чилсло всех матриц в (1.11) равно (С$^{\lambda}_{\alpha}$ -- число сочетаний из $\alpha$ по $\lambda$)\par
\begin{center}
1+С$^{1}_{2\nu}$+С$^{2}_{2\nu}$+\dots+С$^{2\nu}_{2\nu}$=$2^{2\nu}$.\par
\end{center}
Так как система из 2$^{2\nu}$ матриц (1.11) линейно независима, то очевидно, что минимальный порядок матриц $\gamma_i$, удовлетворяющих уравнению (1.1), равен $2^\nu$ (в этом случае число элементов матриц $\gamma_i$ равно $2^{2\nu}$). Очевидно также, что существуют решения уравнения (1.1) в виде матриц $\gamma_i$ порядка $2^x$, x>$\nu$ -- целое положительное число. Такими решениями вида являются, например, квазидиагональные матрицы вида\par
\begin{center}
$$
\gamma=\begin{vmatrix}
\gamma_i & 0 &\ldots & 0\\
0 & \gamma_i &\ldots & 0\\
\hdotsfor{5}\\
0 & 0 &\ldots &\gamma_i
\end{vmatrix}
$$
\end{center}\par
где $\gamma_i$ -- матрицы порядка $2^{\nu}$, удовлетворяющие уравнению (1.1), 0 -- нулевая квадратная матрица порядка $2^{\nu}$. Можно показать, что все решения уравнения (1.1) исчерпываются матрицами, подобными $\gamma_i$, т.е. имеют вид T$_{\gamma_i}$T$^{-1}$, где Т -- произвольная невырожденная матрица.\par
Отметим, что все свойства матриц $\gamma_i$, удовлетворяющих уравнению (1.1), сформулированные в пунктах 1-5, не\par
зависят от порядка матриц $\gamma_i$\.par
Ясно, что если матрицы $\gamma_i$ (i=1,2,\dots ,$2\nu$), удовлетворяющие уравнению (1.1), имеют порядок $2^\nu$, то система матриц (1.11) образует базис в полной матричной алгебре над полем комплексных чисел, размерность которой равна $2^{2^\nu}$.
6. Ввиду полноты и линейной независимости системы матриц (1.11) любая комплексная матрица $\psi$ порядка $2^\nu$ может быть представлена в виде
\begin{center}
$\psi = \frac{1}{2^\nu}(FI+\sum_{k=1}^{2\nu}\frac{1}{k!}F^{i_1,i_2,\dots,i_k}\gamma_{i_1,i_2,\dots,i_k})$   (1.14)\par
\end{center}
Для того чтобы определить коэффициенты F,$F^{i_1,i_2,\dots,i_k}$, входящие в формулу (1.14), умножим формулу (1.14) на $\gamma_{i_1,i_2,\dots,i_k}$ и возьмем след от получившегося выражения. В результате получим\par
tr($\psi_{\gamma^{j_1,j_2,\dots,j_m}})=\frac{1}{2^\nu}[Ftr\gamma^{j_1,j_2,\dots,j_m}+$\par
\begin{center}
+$\sum_{k=1}^{2\nu}\frac{1}{k!}F^{i_1,i_2,\dots,i_k}tr(\gamma_{i_1,i_2\dots i_k}\gamma^{j_1,j_2\dots j_m})]$.   (1.15)
\end{center}\par
Отсюда, пользуясь равенствами (1.10), для коэффициентов $F^{i_1,i_2,\dots,i_k}$ найдем\par
\begin{center}
$F^{i_1,i_2,\dots,i_k}=(-1)^{\frac{1}{2}k(k-1)}tr(\psi_{\gamma^{j_1,j_2,\dots,j_m}}).$(1.16)\par
\end{center}
Здесь$^1$)\par
\begin{center}
$\gamma^{i_1,i_2\dots i_k}=\gamma_{i_1,i_2\dots i_k}.$ (1.17)\par
\end{center}
Беря след от равенства (1.14), найдем коэффициент F\par
\begin{center}
F=tr$\psi$ (1.18)
\end{center}
В частности, в качестве $\psi$ в равенстве (1.14) можно взять произведение любых матриц $\gamma_{i_1,i_2\dots i_k}$. В этом случае равенство (1.14) будем записывать в виде\par
$\gamma_{i_1,i_2\dots i_k}\gamma_{j_1,j_2\dots j_m}$=\par
\begin{center}
=$A_{i_1 i_2\dots i_k j_1 j_2\dots j_m}I+\sum_{q=1}^{2\nu}A_{i_1\dots i_k j_1\dots j_m}^{s_1 \dots s_q}\gamma_{s_1\dots s_q}$. (1.19)\par
\end{center}
Согласно определениям (1.16),(1.18), для коэффициентов A имеем\par
$A_{i_1 \dots i_k j_1 \dots j_m}=\frac{1}{2\nu}tr(\gamma_{i_1 \dots i_k}\gamma_{j_1\dots j_m})$,\par
$A_{i_1\dots i_k j_1\dots j_m}^{s_1 \dots s_q}=(-1)^{\frac{1}{2}q(q-1)}\frac{1}{2\nu}tr(\gamma_{i_1 \dots i_k}\gamma_{j_1\dots j_m}\gamma^{s_1\dots s_q}).$(1.20)\par
Коэффициенты А представляют собой сумму разилчных произведений символов Кронекера. Прямое вычисление показывает, что формулу (1.19) можно записать в виде[17]\par
$\gamma_{i_1 \dots i_k}\gamma_{j_1 \dots j_m}$=\par
=$\sum_{p=0}(-1)^{\frac{1}{2}p(2k-p-1)}\frac{k!m!}{p!(k-p)!(m-p)!}\delta_{s_1 q_1} \dots \delta_{s_p q_p}\times$\par
$\times\delta_{[i_1}^{s_1}\dots\delta_{i_p}^{s_p}\delta_{i_{p+1}}^{s_{p+1}}\dots\delta_{i_k]}^{s_k}\delta_{[j_1}^{q_1}\dots\delta_{j_p}^{q_p}\delta_{j_{p+1}}^{q_{p+1}}\dots\delta_{j_m]}^{q_m}\gamma_{s_{p+1}\dots s_k q_{p+1} \dots q_m}$ (1.21)\par
\footnote[1]{В дальнейшем индексы $i_k$ в $\gamma$-матрицах рассматриваются как тензорные индексы в евклидовом пространстве с метрическим тензором, определяемым в ортонормированном базисе компонентами $\delta_{ij}$. Здесь пока формально используется запись индексов $i_k$ на разных местах с тем, чтобы получаемые уравнения были ковариантными тензорными уравнениями.}\nopagebreak\par
Здесь\par
$\lambda=min(k,m),$\par
$$
\theta=\begin{cases}
0,&\text{если $\frac{1}{2}(k+m-2\nu+1)\le0$;}\\
[\frac{1}{2}(k+m-2\nu+1)],&\text{если $\frac{1}{2}(k+m-2\nu+1)>0.$.}
\end{cases}
$$
\par
\hspace{0.2cm}
Скобки [] в формуле для $\theta$ означают здесь целую часть от числа, заключенного в скобки. Суммирование по p в (1.21) обрывается, если k+m-2p>$2\nu$. Для простоты записи в (1.21) индексы с нулевым номером опускается и принимается, что $\gamma_{s_0}=\gamma_{q_0}$=I.\par
Отметим, что правая часть равенства (1.21) содержит единичную матрицу только в случае, если k=m.\par
В различных вычислениях с $\gamma$-матрицами удобно пользоваться также другой записью формулы (1.21):\par
$\gamma_{i_1 \dots i_k}\gamma^{j_1 \dots j_m}=\sum_{p=0}^{\lambda}(-1)^{\frac{1}{2}p(2k-p-1)}\frac{k!m!}{p!(k-p)!(m-p)!}\times$\par
\begin{center}
$\times\delta_{[i_1}^{[j_1}\dots\delta_{i_p}^{j_p}\gamma_{i_{p+1}\dots i_k]}^{j_{p+1}\dots j_m]}.$(1.22) \par
\end{center}
Здесь обозначено\par
\begin{center}
$\gamma_{i_{p+1}\dots i_k}^{j_{p+1}\dots j_m}=\gamma_{i_{p+1} \dots i_{kj_{p+1}} \dots j_m}$.(1.23)
\end{center}\par
Приведем используемые далее уравнения (1.22):\par
при m=1\par
\begin{center}
$\gamma_{i}\gamma_{j}=\gamma_{ij}+\delta_{ij}I,$\par
$\gamma_{i_1 i_2 \dots i_k}\gamma_{j}=\gamma_{i_1 i_2 \dots i_k j}+k(-1)^{k-1}\delta_{j[i_1\gamma_{i_2 \dots i_k]}}\,$\par
$k=2,3,\dots,2\nu-1,$\par
$\gamma_{i_1 i_2 \dots i_{2\nu}}\gamma_{j}=-2\nu\delta_{j[i_1\gamma_{i_2 \dots i_{2\nu}}]}$\par
\end{center}
при k=1\par
$\gamma_{j}\gamma_{i_1 \dots i_m}=\gamma_{j i_1 \dots i_m}+m\delta_{j[i_1\gamma_{i_2 \dots i_m}]}, m=2,3,\dots,(2-1),$\par
$\gamma_{j}\gamma_{i_1 i_2 \dots i_{2\nu}}=2\nu\delta_{j[i_1\gamma_{i_2 \dots i_{2\nu}}]};$ (1.24б)\par
при m=2\par
$\gamma{i_1}\gamma^{j_1 j_2}=\gamma_{i_1}^{j_1 j_2}+2\delta_{i_1}^{[j_1\gamma j_2]},$\par
$\gamma_{i_1 i_2}\gamma^{j_1 j_2}=\gamma_{i_1 i_2}^{j_1 j_2}+4\delta_{[i_1 i_2]}^{[j_1\gamma j_2]}-2\delta_{[i_1 i_2]}^{j_1 \delta j_2}I,$\par
$\gamma_{i_1 i_2 \dots i_k}\gamma^{j_1 j_2}=\gamma_{i_1 i_2 \dots i_k}^{j_1 j_2}+2k\delta_{[i_1 i_2 \dots i_k]}^{[j_1 \gamma j_2]}$-\par
$-k(k-1)\delta^{j_1 \delta j_2}_{[i_1 i_2 \gamma_{i_3 \dots i_k}]}, k=3,4,\dots,2\nu-2,$ (1.24в)\par
$\gamma_{i_1 i_@ |dots i_{2\nu}}\gamma^{j_1 j_2}=-2\nu(2\nu-1)\delta_{[i_1 i_2 \dots i_{2\nu}]}^{j_1 \gamma^{j_2}}-$\par
\begin{center}
$-(2\nu-1)(2\nu-2)\delta_{[i_1\dots i_{2\nu-1}]}^{j_1 \delta_{i_2}^{j_2} \gamma_{i_2}},$\par
\end{center}
$\gamma_{i_1 i_2 \dots i_{2\nu}}\gamma^{j_1 j_2}=-2\nu(2\nu-1)\delta_{[i_1 \dots i_{2\nu}}^{j_1 \delta_{i_2}^{j_2}\gamma_{i_2}};$\par
при k=2\par
$\gamma_{i_1 i_2}\gamma^{j_1}=\gamma_{i_1 i_2}^{j_1}-2\delta_{[i_1}\gamma_{i_2]}^{j_1},$\par
$\gamma_{i_1 i_2}\gamma^{j_1 j_2 \dots j_m}=\gamma_{i_1 i_2}^{j_1 j_2 \dots j_m}-2m\delta^{[j_1 j_2 \dots j_m]}_{[i_1 \gamma_{i_2}]}$-\par
$-m(m-1)\delta_{[i_1}^{[j_1}\delta_{i_2}^{j_2}\gamma^{j_3 \dots j_m]}, m=3,4,\dots,2\nu-2, (1.24г)$\par
$\gamma_{i_1 i_2}\gamma^{j_1 j_2 \dots j_{2\nu-1}}=-2(2\nu-1)\delta_{[i_1}^{[j_1}\gamma_{i_2]}^{j_2 \dots j_{2\nu-1}]}$-\par
\begin{center}
$-(2\nu-1)(2\nu-2)\delta_{i_1}^{[j_1}\delta_{i_2}^{j_2}\gamma^{j_3 \dots j_{2\nu-1}]},$\par
\end{center}
$\gamma_{i_1 i_2}\gamma^{j-1 j_2 \dots j_{2\nu}}=-2\nu(2\nu-1)\delta_{i_1}^{[j_1}\delta_{i_2}^{j_2}\gamma^{j_3 \dots j_{2\nu}]};$\par
\hspace{0.2cm}
при m=$2\nu$ \par
$\gamma_{i_1 i_2 \dots i_{2\nu}}\gamma^{j_1 j_2 \dots j_{2\nu}}=(-1)^{\nu}(2\nu)\delta_{i_1}^{[j_1}\delta_{i_2}^{j_2}\dots\delta_{i_{2\nu}}^{j_{2\nu}]}I,$\par
$\gamma_{i_1 i_2 \dots i_k}\gamma^{j_1 j_2 \dots j_{2\nu}}=(-1)^{\frac{1}{2}k(k-1)}\frac{(2\nu)!}{2\nu-k)!}\delta_{i_1}^{[j_1}\delta_{i_2}^{j_2}\dots$\par
\begin{center}
$\dots\delta_{i_k}^{j_k}\gamma^{j_{k+1}\dots j_{2/nu}]}, k=1,2,\dots,2\nu-1;$\par
\end{center}
при k=$2\nu (m=1,2,\dots,2\nu-1)$\par
$\gamma_{i_1 i_2 \dots i_{2\nu}}\gamma^{j_1 j_2 \dots j_m}=$\par
\hspace{0.2cm}
$=(-1)^{\frac{1}{2}m(m+1)}\frac{(2\nu)!}{(2\nu-m)!}\delta_{[i_1}^{j_1}\delta_{i_2}^{j_2}\dots\delta_{i_m}^{j_m}\gamma_{i_{m+1}\dots i_{2\nu}]}.$(1.24e)\par
7.Система матриц, состоящая из произведений любой матрицы в системе (1.11) на каждую матрицу в (1.11), содержит в себе, с точностью до знака, все матрицы (1.11).\par
Действительно, из невырожденности и линейной независимости матриц (1.11) следует, что рассматриваемая система из произведений $\gamma$-матриц также линейно независима и содержит поэтому $2^{2\nu}$ различных матриц. Из формулы (1.22) ( или, проще, непосредственно из определения (1.3) и из уравнения (1.1)) следует, что эти произведения представляют собой (во всяком случае, с точностью до знака) одну из матриц в (1.11).\par
8. Тождество Паули. Обозначим элементы матриц\par
$\gamma_{i_1 \dots i_k}, \gamma^{i_1 \dots i_k}$, соответственно символами $\gamma_{Ai_1\dots i_k}^{B}, \gamma_{A}^{Bi_1 \dots i_k}$, а элементы матриц $\psi$ в формуле (1.14) -- символом $\psi_{A}^{B}$.\par
\begin{center}
$\psi=||\psi_{A}^{B}||, \gamma_{i_1 \dots i_k}=||\gamma_{Ai_1 \dots i_k}^{B}||, \gamma^{i_1 \dots i_k}=||\gamma_{A}^{Bi_1 \dots i_k}||,$(1.25)\par
\end{center}
причем первый индекс, B, в элементах матрицы $\psi$ и матриц $\gamma$ означает номер строки, второй индекс, А, означает номер столбца.\par
C помощью введенных обозначений формулу (1.14) можно записать в виде\par
\begin{center}
$\psi_{A}^{B}=\frac{1}{2^{\nu}}(F\delta_{A}^{B}+\sum_{k=1}^{2\nu}\frac{1}{k!}F^{i_1 i_2 \dots i_k}\gamma^{B}_{Ai_1 i_2 \dots i_k})$,(1.26) \par
\end{center}
а определения (1.16), (1.18) для коэффициентов F, $F^{i_1 i_2 \dots i_k}$ - в виде\par
\begin{center}
$F=\psi_{A}^{A},$\par
$F^{i_1 i_2 \dots i_k} = (-1)^\frac{1}{2}k(k-1)\gamma_{A}^{Bi_1 i_2 \dots i_k}\psi_{B}^{A}.$\par
\end{center}
Внесем в равенство (1.26) коэффициенты F, $F^{i_1 i_2 \dots i_k}$ согласно определениям (1.27):\par
$[-\delta_{A}^{C}\delta_{D}^{B}+\frac{1}{2^{\nu}}(\delta_{A}^{C}\delta_{D}^{B}+$ \par
\begin{center}
$+\sum_{k=1}^{2\nu}\frac{1}{k!}(-1)^{\frac{1}{2}k(k-1)}\gamma^{C i_1 \dots i_k}_{D}\gamma^{B}_{A i_1 \dots i_k})]\psi^{D}_{C}=0.$(1.28)\par
\end{center}
Из этого равенства в силу произвольности величин $\psi^{D}_{C}$ получим следующее важное тождество, связывающее произведения $\gamma$-матриц:\par
$\delta_{A}^{C}\delta_{D}^{B}=\frac{1}{2^{\nu}}(\delta_{A}^{C}\delta_{D}^{B}+\sum_{k=1}^{2\nu}\frac{1}{k!}(-1)\frac{1}{2}k(k-1)\gamma^{C i_1 \dots i_k}_{D}\gamma^{B}_{A i_1 \dots i_k})$. (1.29)\par
Тождество (1.29) при $\nu$=2 было получено Паули [45] и обычно называется тождеством Паули.\par
Умножим тождество (1.29) на $\gamma^{M}_{Cj_1 \dots j_m}\gamma^{E}_{Bs_1 \dots s_q}$ и просуммируем результат по индексам B, C от 1 до 2$\nu$:\par
$\gamma^{M}_{Cj_1 \dots j_m}\gamma^{E}_{Bs_1 \dots s_q}=\frac{1}{2^{\nu}}[\gamma^{M}_{Dj_1 \dots j_m}\gamma^{E}_{As_1 \dots s_q}+$\par
$+\sum_{k=1}^{2\nu}\frac{1}{k!}(-1)^\frac{1}{2}k(k-1)(\gamma^{M}_{Cj_1 \dots j_m}\gamma^{Ci_1 \dots i_k}_{D})(\gamma^{E}_{Bs_1 \dots s_q}\gamma^{B}_{Ai_1 \dots i_k}]$. (1.30) \par
Заменяя здесь произведения матриц $\gamma$ по формуле (1.19), получим соотношения, выражающие произведения $\gamma^{M}_{A}\gamma^{E}_{D}$ через сумму различных произведений $\gamma^{M}_{D}\gamma^{E}_{A}$ с переставленными индексами D, A:\par
$\gamma^{M}_{Аj_1 \dots j_m}\gamma^{E}_{Ds_1 \dots s_q}=\sum_{k=0}^{2\nu}\sum_{p=0}^{2\nu}\alpha_{j_1 \dots j_m s_1 \dots s_q}^{i_1 \dots i_k l_1 \dots l_p}\gamma^{M}_{D i_1 \dots i_k}\gamma^{E}_{A l_1 \dots l_p}. (1.31)$ \par
Здесь коэффициенты $\alpha$ определяются равенствами \par
$\alpha_{j_1 \dots j_m s_1 \dots s_q}^{i_1 \dots i_k l_1 \dots l_p}=\sum_{r=0}^{2\nu}\frac{1}{{2^\nu}r!}(-1)\frac{1}{2}r(r-1)\delta^{t_1 n_1}\dots $\par
\begin{center}
$\dots \delta^{t_r n_r}A_{j_1 \dots j_m t_1\dots t_r}^{i_1 \dots i_k}A_{s_1 \dots s_q n_1 \dots n_r}^{l_1 \dots l_p}.$ (1.32)\par
\end{center}
Пользуясь соотношениями (1.10), нетрудно найти, что для коэффициентов $\alpha$ справедливо также следующее вы-\par
ражение: \par
$\alpha_{j_1 \dots j_m s_1 \dots s_q}^{i_1 \dots i_k l_1 \dots l_p}=(-1)^{\frac{1}{2}[k(k-1)+p(p-1)]}\frac{1}{2^{2\nu}k!p!}\times$\par
\begin{center}
$\times tr(\gamma^{i_1 \dots i_k}\gamma_{j_1 \dots j_m}\gamma^{l_1 \dots l_p}\gamma_{s_1 \dots s_q}).$ (1.33)\par
\end{center}
9.Введем матрицу $\gamma_{2\nu+1}$:
\begin{center}
$\gamma_{2\nu+1}=i^{\nu}\gamma_{1}\gamma_{2}\dots\gamma_{2\nu}.$\par (1.32)
\end{center}
Система матриц с четным числом индексов\par
\begin{center}
I, $\gamma_{i_1 i_2} \dots, \gamma_{i_1 i_2 \dots i_{2\nu}} (i_1<i_2< \dots<i_{2\nu})$ (1.35)\par
\end{center}
и система матриц с нечетным числом \par
\begin{center}
$\gamma_i,\gamma_{i_1 i_2 i_3},\dots ,\gamma_{i_1 i_2 i_3,\dots i_{2\nu+1}} (i_1<i_2<\dots<i_{2\nu+1}),$ (1.36)\par
\end{center}
в которых $\gamma_{i_1 \dots i_k}=\gamma_{[i_1} \dots \gamma_{i_k]}$, индексы $i_k$ принимают\par
значения от 1 до $2\nu$+1, являются линейно независимыми.\par
Доказательство линейной независимости систем матриц  (1.35), (1.36) аналогично соответствующему доказательству для систем матриц (1.11) в п. 5.\par

Очевидно, что число матриц в системе (1.35) равно $2^{2\nu}$:\par
\begin{center}
$C^{0}_{2\nu+1}+C^{2}_{2\nu+1}+\dots +C^{2\nu}_{2\nu+1}=2^{2\nu}$.\par
\end{center}
Число матриц в системе (1.36) также равно $2^{2\nu}$:\par
\begin{center}
$C^{1}_{2\nu+1}+C^{3}_{2\nu+1}+ \dots +C^{2\nu+1}_{2\nu+1}=2^{2\nu}.$ \par
\end{center}
Поэтому система матриц (1.35) и система матриц (1.36) образуют базисы в полной матричной алгебре размерности $2^{2\nu}$ \par
Ввиду полноты и линейной независимости системы матриц (1.35) любая комплексная матрица $||\psi^{B}_{A}||$ порядка $2\nu$ может быть записана в виде: \par
$\psi^{B}_{A}=\frac{1}{2^{\nu}}(F\delta^{B}_{A}+\sum_{k=1}^{\nu}\frac{1}{(2k)!}F^{i_1 i_2 \dots i_2k}\gamma^{B}_{Ai_1 i_2 \dots i_2k}).$(1.37)\par
Для коэффициентов F, $F^{i_1 i_2 \dots i_2k}$ вычислениями, аналогичными проведенным в п. 6, найдем \par
\begin{center}
$F=\psi^{A}_{A}, F^{i_1 i_2 \dots i_2k}=(-1)^{k}\gamma^{Bi_1 i_2 \dots i_2k}_{A}\psi^{A}_{B}.$     (1.38)\par
\end{center}
Используя систему матриц (1.38), для $\psi^{B}_{A}$ можно написать\par
\begin{center}
$\psi^{B}_{A}=\frac{1}{2\nu}\sum_{k=0}^{\nu}\frac{1}{(2k+1)!}F^{i_1 i_2 \dots i_2k+1}\gamma^{B}_{Ai_1 i_2 \dots i_2k+1},$     (1.39) \par
\end{center}
где\par
\begin{center}
$F^{i_1 i_2 \dots i_2k+1}=(-1)^{k}\gamma^{Bi_1 i_2 \dots i_2k+1}\psi^{A}_{B}.$ (1.40)\par
\end{center}
10. Если матрица А коммутирует со всеми матрицами \par
$\gamma_i$:\par
\begin{center}
A$\gamma_{i}=\gamma_{i}$A, (1.41) \par
\end{center}
то матрица А кратна единичной матрице:\par
\begin{center}
A=$\lambda$I.      (1.42) \par
\end{center}
Здесь $\lambda$ - некоторое, в общем случае комплексное, число.\par
Для доказательства представим матрицу А в виде \par
\begin{center}
A=$\alpha$I+$\sum_{k=1}^{2\nu}\alpha^{i_1 \dots i_k}\gamma_{i_1 \dots i_k},$      (1.43)\par
\end{center}
где коэффициенты $\alpha^{i_1 \dots i_k}$ антисимметричны по всем индексам. Имеем \par
\begin{center}
$\gamma_{j}A-A\gamma_{j}=\sum_{k=1}^{2\nu}\alpha^{i_1 \dots i_k}(\gamma_{j}\gamma_{i_1 \dots i_k}-\gamma_{i_1 \dots i_k}\gamma_{j}).$\par
\end{center}
Заменяя здесь разность произведений $\gamma$-матриц по формулам\par
\begin{center}
$\gamma_{j}\gamma_{i_1}-\gamma_{i_1}\gamma_{j}=2\gamma_{ji_1},$\par
\end{center}
$\gamma_{j}\gamma_{i_1 \dots i_k}-\gamma_{i_1 \dots i_k}\gamma_{j}=[1-(-1)^{k}]\gamma_{j}\gamma_{i_1 \dots i_k}+$\par
$+k[1+(-1)^{k}\delta_{j[i_1}\gamma_{i_2 \dots i_k]}, k=2,3,\dots,2\nu-1,$ (1.44)\par
$\gamma_{j}\gamma_{i_1 i_2\dotsi_{2\nu}}-\gamma_{i_1 i_2\dotsi_{2\nu}}\gamma_{j}=4\nu\delta_{j[i_1}\gamma_{i_2 \dots i_{2\nu}]},$\par
следующим из соотношений (1.24а, б) получим\par
$\gamma_{j}A-A\gamma_{j}=2\sum_{m=0}^{\nu-1}a^{i_1 \dots i_{2m+1}}\gamma_{ji_1\dots i_{2m+1}}+ $\par
\begin{center}
$+4\sum_{m=1}^{\nu}a^{i_1 i_2 \dots i_{2m}}m\delta_{j]i_1}\gamma_{i_2 \dots i_{2m}]}=0.     (1.45)$\par
\end{center}
Первая сумма в (1.45) содержит $\gamma$-матрицы только с четным числом индексов, вторая - только с нечетным числом индексов.\par
В силу линейности независимости системы матриц (1.11) из уравнения (1.45) следует, что все коэффициенты $a^{i_1 \dots i_k}$ равны нулю. Таким образом, в разложении (1.43) отличен  от нуля только член с единичной матрицей.\par
\textbf{11. }Если матрицы $\gamma_i$ удовлетворяют уравнению (1.1) и \textsl{T} - произвольная невырожденная матрица порядка $2^\nu$, то очевидно, что система матриц $\gamma_i$,\par
\begin{center}
$\gamma_{i}=\textsl{T}^{-1}\gamma_{i}\textsl{T},$     (1.46)\par
\end{center}
также удовлетворяют уравнению \par
\begin{center}
$\gamma_{i}\gamma_{j}+\gamma_{j}\gamma_{i}=2\delta_{ij}\textsl{I}.$\par
\end{center}
\hspace{0.2cm}
Оказывается, что и любые две системы матриц $\gamma_{i},\gamma_{i}$, удовлетворяющие уравнению (1.1), всегда связаны соотношением (1.46), причем матрица \textsl{T} определяется с точностью до умножения на произвольное комплексное число, не равное нулю. Сформулированное свойство решений уравнения (1.1) носит название теоремы Паули.\par
Простое доказательство теоремы Паули состоит в явном указании матрицы \textsl{T}, соответствующей различным наборам матриц $\gamma_{i},\gamma_{i}$. Если обозначить матрицы (1.11) символом $\gamma_{A}$ (A=1,2,\dots,$2^{\nu}$), а такие же матрицы, образованные из $\gamma_{i}$ символом $\gamma_{A}$, то матрицу \textsl{T} можно записать в виде\par
\begin{center}
$\textsl{T}=\sum_{A=1}^{2^{2\nu}}\gamma_{A}F\gamma^{-1}_{A},$     (1.47)\par
\end{center}
Где \textsl{F} - некоторая ненулевая квадратная матрица порядка $2^{\nu}$.\par
Действительно, вычислим величину $\gamma_{i}\textsl{T}\gamma^{-1}_{i}$ (по индексу i суммирование не производится):\par
\begin{center}
$\gamma_{i}\textsl{T}\gamma^{-1}_{i}=\sum_{A=1}^{2^{2\nu}}\gamma_{i}\gamma_{A}\textsl{F}(\gamma_{i}\gamma_{A})^{-1}. $   (1.48)\par
\end{center}
Так как согласно свойству 7 матриц $\gamma_{A}$ произведение  $\gamma_{i}\gamma_{A}$ при всех \textsl{A} снова дает все матрицы $\gamma_{A}$ (во всяком случае, с точностью до знака), то равенство (1.48) можно продолжить:\par
\begin{center}
$\gamma_{i}\textsl{T}\gamma^{-1}_{i}=\sum_{A=1}^{2^{2\nu}}\gamma_{A}\textsl{F}\gamma^{-1}_{A}=\textsl{T}.$\par
\end{center}
Отсюда получаем\par
\begin{center}
$\tilde\gamma_{i}\textsl{T}=\textsl{T}\gamma_{i}$.      (1.49)\par
\end{center}
Очевидно, что всегда можно выбрать матрицу \textsl{F} в определения (1.47) таким образом, чтобы $\textsl{T}\ne0$ (в противном случае система матриц $\gamma_{A}$ была бы линейно зависимой). Покажем, что при соответствующем выборе \textsl{F} матрица \textsl{T} невырождена, $det\textsl{T}\ne0$.\par
Аналогично (1.47)-(1.49) получается, что матрица \textsl{Q}, определенная равенством\par
\begin{center}
$Q=\sum_{A=1}^{2^{2\nu}}\gamma_{A}\textsl{G}\tilde\gamma_{A}^{-1}$,\par
\end{center}
в котором \textsl{G} - некоторая квадратная матрица порядка $2^{\nu}$, удовлетворяет уравнению\par
\begin{center}
$\gamma_{i}\textsl{Q}=\textsl{Q}\tilde\gamma_{i}$       (1.50)\par
\end{center}
и при соответствующем выборе \textsl{G} отлична от нуля, $\textsl{Q}\ne0.$ Умножая уравнение (1.50) справа на \textsl{T}  с учетом уравнения (1.49), найдем\par
\begin{center}
$\gamma_{i}\textsl{QT}=\textsl{QT}\gamma_{i},$\par
\end{center}
и значит, матрица \textsl{QT} пропорциональна единичной матрице:\par
\begin{center}
$\textsl{QT}=\alpha\textsl{I}.$       (1.51) \par
\end{center}
Из линейной независимости системы матриц $\gamma_{A}$ следует, что при $\textsl{Q}\ne0$ всегда можно выбрать \textsl{F} таким образом, что число $\alpha$  в уравнении (1.51) отлично от нуля ($\alpha\ne0$). В самом деле, если при любой матрицу \textsl{F} было бы $\alpha=0$, то из уравнения (1.51) следовало бы равенство\par
\begin{center}
$\sum_{A=1}^{2^{2\nu}}(\textsl{Q}\tilde\gamma_{A}\textsl{F})\gamma_{A}^{-1}=0$\par
\end{center}
и, в силу произвольности \textsl{F},\par
\begin{center}
$\sum_{A=1}^{2^{2\nu}}(\textsl{Q}\tilde\gamma_{A})\times\gamma^{-1}_{A}=0$.        (1.52)\par
\end{center}
Но так как $\textsl{Q}\ne0$, то все матрицы $\textsl{Q}\tilde\gamma_{A}$ не могут быть равными нулю. Поэтому равенство (1.52) противоречит линейной независимости системы матриц $\gamma_{A}$. Значит, всегда можно выбрать матрицу \textsl{F} такую, что $\alpha\ne0$.\par
Из уравнения (1.51) следует, что при соответствующем выборе \textsl{F} матрица \textsl{T} невырождена и существует обратная матрица $\textsl{T}^{-1}=\alpha^{-1}\textsl{Q}.$ Поэтому из равенства (1.49) следует доказываемое уравнение (1.46).\par
Осталось показать, что матрица \textsl{T} в уравнении (1.46) определения с точностью до умножения на произвольное ненулевое число. Предположим, что \textsl{\~T}$\ne$\textsl{T} удовлетворяет уравнению\par
\begin{center}
$\tilde\gamma_{i}=\textsl{\~T}\gamma_{i}\textsl{\~T}^{-1}.$ (1.53)\par
\end{center}
Тогда из уравнений (1.46), (1.53) следует\par
\begin{center}
$\textsl{T}^{-1}\textsl{\~T}\gamma_{i}=\gamma_{i}\textsl{T}^{-1}\textsl{~T}$,\par
\end{center}
поэтому\par
\begin{center}
$\textsl{T}^{-1}$\textsl{~T}=$\lambda$\textsl{I}, $\lambda\ne0$,\par
\end{center}
и, значит, \textsl{~T} отличается от \textsl{T} только численным множителем \textsl{~T}=$\lambda$\textsl{T}, $\lambda\ne0$. Теорема доказана.\par
заменяя в уравнении (1.46) матрицы $\gamma_{i}$ по формуле \par
\begin{center}
$\gamma_{i}=-\gamma^{-1}_{2\nu+1}\gamma_{i}\gamma_{2\nu+1},$\par
\end{center}
следующей из определения (1.34) для $\gamma_{2\nu+1}$ найдем, что связь между $\gamma_{i},\tilde\gamma_{i}$ можно записать также в виде\par
\begin{center}
$\gamma_{i}=\textsl{-\.T}^{-1}\gamma_{i}\textsl{\.T},$\par
\end{center}
где\par
\begin{center}
$\textsl{\.T}=\gamma_{2\nu+1}\textsl{T}.$\par
\end{center}
Покажем теперь, что решение матричных уравнений (1.1) для $\gamma_{i}$ при любом $\nu\ge1$ существует и может быть реализовано в виде эрмитовых матриц\par
\begin{center}
$(\gamma_{i})^{T}=(\gamma_{i})^{.},$        (1.54)\par
\end{center}
причем $\nu$ матриц $\gamma_1,\gamma_2,\dots,\gamma_{\nu}$ могут быть выбраны симметричными, а $\nu$ матриц $\gamma_{\nu+1},\gamma_{\nu+2},\dots,\gamma_{2\nu}$ - антисимметричными:\par
\begin{center}
$(\gamma_1)^{T}=\gamma_1, (\gamma_2)^{T}=\gamma_2,\dots, (\gamma_{\nu})^{T}=\gamma_{\nu}$,\par
$(\gamma_{\nu+1})^{T}=-\gamma_{\nu+1}, (\gamma_{\nu+2})^{T}=-\gamma_{\nu+2},\dots, (\gamma_{2\nu})^{T}=-\gamma_{2\nu}.$        (1.55)\par
\end{center}
\hspace{0.2cm}
В равенствах (1.54), (1.55) символ \glqq т \grqq означает транспонирование, точка означает комплексное сопряжение.\par
Докажем сформулированное утверждение.
Для $\nu$=1 легко указать систему двух матриц второго порядка, удовлетворяющих уравнению (1.1):\par
\begin{center}
$$
\gamma_1=\begin{vmatrix}
1 & 0\\
0 & -1
\end{vmatrix},
\gamma_2=\begin{vmatrix}
0 & i\\
-i & 0
\end{vmatrix} (1.56)
$$
\end{center}
Очевидно, что выписанные матрицы эрмитовы, причем $\gamma_1$ симметрична, $\gamma_2$ антисимметрична.\par
Полагая, что решение уравнения (1.1) при $\nu=\alpha$, удовлетворяющее условиям (1.54), (1.55), существует, покажем, что решение уравнения (1.1) существует при $\nu=\alpha+1$ и удовлетворяет условиям (1.54), (1.55).\par
Пусть $\gamma_1,\gamma_2,\dots,\gamma_{2\alpha}$ - система из $2\alpha$ эрмитовых матриц, удовлетворяющих уравнению (1.1), в котором индексы i, j принимают значения от 1 до 2$\alpha$. будем считать, что матрицы $\gamma_1,\gamma_2,\dots,\gamma_{2\alpha}$ симметричны, а $\gamma_{\alpha+1},\gamma_{\alpha+2},\dots,\gamma_{2\alpha}$ антисимметричны. Введем матрицу\par
\begin{center}
$\gamma_{2\alpha+1}=i^{\alpha}\gamma_1,\gamma_2,\dots,\gamma_{2\alpha}.$        (1.57)\par
\end{center}
В силу уравнения (1.1) матрица $\gamma_{2\alpha+1}$ является единичной матрицей: \par
\begin{center}
$\gamma_{2\alpha+1}\gamma_{2\alpha+1}=\textsl{I}$       (1.59)\par
\end{center}
Если матрицы $\gamma_i$ эрмитовы, то матрица $\gamma_{2\alpha+1}$, определенная равенством (1.57), также эрмитова. В самом деле, в силу эрмитовости матриц $\gamma_i$ (i=1,2,$\dots,2\alpha$), для транспонированной матрицы $\gamma^{T}_{2\alpha+1}$ можно написать \par
$\gamma^{T}_{2\alpha+1}=i^{\alpha}\gamma^{T}_{2\alpha}\dots\gamma^{T}_{2}\gamma^{T}_{1}=i^{\alpha}(\gamma_{2\alpha})^{.}\dots(\gamma_{2})^{.}(\gamma_1)^{.}=$\par
\begin{flushright}{$=i^{\alpha}(\gamma_{2\alpha}\dots\gamma_2\gamma_1)^{.}.$ (1.60)}
\end{flushright}\par
Так как все $\gamma_i$ в произведении (1.57) различны, то, согласно уравнению (1.1), все они антикоммутируют между собой. Переставляя $\gamma_i$ в равенстве (1.60), продолжим его\par
\begin{center}
$\gamma^{T}_{2\alpha+1}=i^{\alpha}(-1)^{\alpha(2\alpha-1)}(\gamma_1\gamma_2\dots\gamma_{2\alpha})^{.}=(i^{\alpha}\gamma_1\gamma_2\dots\gamma_{2\alpha})^{.}$.\par
\end{center}
Таким образом,\par
\begin{center}
$\gamma^{T}_{2\alpha+1}=(\gamma_{2\alpha+1})^{.}.$ (1.61)\par
\end{center}
Выясним свойства симметрии матрицы $\gamma_{2\alpha+1}$. Имеем\par
\begin{center}
$\gamma^{T}_{2\alpha+1}=i^{\alpha}\gamma^{T}_{2\alpha}\dots\gamma^{T}_2\gamma^{T}_1=i^{\alpha}(-1)^{\alpha}\gamma_{2\alpha}\dots\gamma_2\gamma_1.$ (1.62)\par
\end{center}
Переставляя в этом равенстве антикоммутирующие матрицы $\gamma_i$ найдем\par
\begin{center}
$\gamma^{T}_{2\alpha+1}=i^{\alpha}(-1)^{\alpha}(-1)^{\alpha(2\alpha-1)}\gamma_1\gamma_2\dots\gamma_{2\alpha}=i^{\alpha}\gamma_1\gamma_2\dots\gamma_{2\alpha}.$ \par
\end{center}
Таким образом, матрица $\gamma_{2\alpha+1}$ симметрична:\par
\begin{center}
$\gamma^{T}_{2\alpha+1}=\gamma_{2\alpha+1}.$        (1.63)\par
\end{center}
Рассмотрим следующую систему из 2$(\alpha+1)$ матриц.\par
\begin{center}
$$
\begin{vmatrix}
0 & I\\
I & 0
\end{vmatrix},
\begin{vmatrix}
0 & -i\gamma_j\\
i\gamma_j & 0
\end{vmatrix},
$$
j=1,2,\dots,2$\alpha+1$.  (1.64)\par
\end{center}
\par
Непосредственная проверка с помощью уравнений (1.58), (1.59) показывает, что система 2($\alpha+1)$ матриц (1.64) удовлетворяет уравнению (1.1), в котором i,j=1,2\dots,2$(\alpha+1)$.\par
Установим свойства симметрии системы матриц (1.64). Очевидно, что первая матрица в (1.64) эрмитова и симметрична. Остальные матрицы в (1.64) также эрмитовы:\par
$$
\begin{vmatrix}
0 & -i\gamma_j\\
i\gamma_j & 0
\end{vmatrix}=
\begin{vmatrix}
o & i\gamma^{T}_j\\
-i\gamma^{T}_j & 0
\end{vmatrix}=
\begin{vmatrix}
0 & i(\gamma_j)^{.}\\
-i(\gamma_j)^{.} & 0
\end{vmatrix}=
\begin{vmatrix}
0 & i(\gamma_j)^{.}\\
(i\gamma_j)^{.} & 0
\end{vmatrix}.(1.65)
 $$
 \par
В силу предположенных свойств симметрии матриц $\gamma_j$ (j=1,2\dots,$2\alpha$) и в силу симметричности матрицы $\gamma_{2\alpha+1}$, матрицы (1.64) при j=1,2\dots,$\alpha$ и при j=$2\alpha+1$ антисимметричны:\par
$$
\begin{vmatrix}
0 & i\gamma_j\\
i\gamma_j & 0
\end{vmatrix}=
\begin{vmatrix}
0 & i\gamma^{T}_j\\
-i\gamma^{T}_j & 0
\end{vmatrix}=
\begin{vmatrix}
0 & -i\gamma_j\\
i\gamma_j & 0
\end{vmatrix},(1.66)
$$
а при j=$\alpha+1,\dots,2\alpha$ симметричны:\par
$$
\begin{vmatrix}
0 & -i\gamma_j\\
i\gamma_j & 0
\end{vmatrix},
\begin{vmatrix}
0 & i\gamma^{T}_j\\
-i\gamma^{T}_j & 0
\end{vmatrix}=
\begin{vmatrix}
0 & -i\gamma_j\\
i\gamma_j & 0
\end{vmatrix}. (1.67)$$
Значит, половина матриц в (1.64) симметрична и половина антисимметрична. Таким образом, система матриц (1.64) удовлетворяет уравнению (1.1) и обладает свойствами (1.54), (1.55). Тем самым сформулированное утверждение относительно существования решений уравнения (1.1) доказано.\par
Так как при $\nu$=1 порядок матриц (1.56) равен двум, а при переходе от $\nu=\alpha$ к $\nu=\alpha+1$  в рассмотренном выше построении матриц $\gamma_j$ их порядок удваивается, то очевидно, что порядок $2(\alpha+1)$ матриц (1.64) равен $2^{\alpha+1}$. Выше было показано, что это - минимальный порядок матриц $\gamma_j$, удовлетворяющих уравнению (1.1).\par
Если $\gamma_i$ - некоторое решение уравнения (1.1), то очевидно, что транспонированные матрицы $\gamma^{T}_i$ также являются решением уравнения (1.1):\par
\begin{center}
$\gamma^{T}_i\gamma^{T}_j+\gamma^{T}_j\gamma^{T}_i=2\delta_{ij}I$.        (1.68)\par
\end{center}
Поэтому из теоремы Паули следует, что существует определенная с точностью до умножения на произвольное ненулевое комплексное число невырожденная матрица E=$||e_{BA}||$ такая, что \par
\begin{center}
$\gamma^{T}_i$=-E$\gamma_i E^{-1}.$        (1.69)\par
\end{center}
Согласно теореме Паули два любых решения уравнения (1.1) связаны преобразованием подобия (1.46). Выясним, каким образом меняется матрица \textsl{E}, определяющаяся уравнением (1.69), при переходе от системы матриц $\gamma_i$ к системе матриц $\gamma_i^{'}$\par
\begin{center}
$\gamma_i^{'}=T^{-1}\gamma_i T$.        (1.70) \par
\end{center}
Пусть \textsl{$E^{'}$} определяется уравнением \par
\begin{center}
$(\gamma_i^{'})^{T}=-E^{'}\gamma_i^{'}((E^{'}))^{-1}.$       (1.71)\par
\end{center}
Заменяя в определении (1.71) матрицы $\gamma^{'}_i$ через $\gamma_i$ по формуле (1.70) и умножая полученное равенство слева на $(T^{-1})^{T}$ и справа на $T^{T}$, получим\par
\begin{center}
$\gamma^{T}_i=-(T^{T})^{-1}E^{'}T^{-1}\gamma_i T(E^{'})^{-1}T^{T}.$       (1.72)\par
\end{center}
Сравнивая уравнения (1.69), (1.72), найдем \par
\begin{center}
$\gamma_i E^{-1}(T^{T})^{-1}E^{'}T^{-1}=E^{-1}(T^{T})^{-1}E^{'}T^{-1}\gamma_i.$ (1.73)\par
\end{center}
Отсюда, в силу свойства $\gamma$-матриц, сформулированного в п. 10, следует\par
\begin{center}
$E^{-1}(T^{T})^{-1}E^{'}T^{-1}=\lambda I$,       (1.74)\par
\end{center}
где $\lambda$ - некоторое комплексное число, отличное от нуля. Очевидно, что без ограничения общности можно положить $\lambda=1$ (за счет переопределения \textsl{T}$\to$\textsl{T}$\lambda^{-1/2}$, при котором уравнение (1.70) не меняется), тогда\par
\begin{center}
$E^{'}=T^{T}ET.$       (1.75)\par
\end{center}
Таким образом, если матрицы $\gamma_i\gamma^{'}_i$ связаны преобразованием подобия (1.70), то соответствующие им матрицы E, $E^{'}$ связаны равенством (1.75).\par
Транспонируем уравнение (1.69) и умножим результат справа на $(E^{T})^{-1}$ и слева на $E^{T}$. Получим \par
\begin{center}
$\gamma^{T}_i=-E^{T}\gamma_i(E^{T})^{-1}.$       (1.76)\par
\end{center}
Сравнивая уравнения (1.76), (1.69), находим \par
\begin{center}
$E\gamma_i E^{-1}=E^{T}\gamma_i(E^{T})^{-1}$.       (1.77)\par
\end{center}
Из последнего уравнения следует, что матрица $E^{-1}E^{T}$ коммутирует со всеми $\gamma_i$:\par
\begin{center}
$E^{-1}E^{T}\gamma_i=\gamma_i E^{-1}E^{T}$,       (1.78)\par
\end{center}
и поэтому пропорциональна единичной матрице:\par
\begin{center}
$E^{-1}E^{T}=\rho\textsl{I}$     (1.79)\par
\end{center}
или\par
\begin{center}
$E^{T}=\rho E$.     (1.80)\par
\end{center}
Здесь $\rho$ - некоторое вообще комплексное число. Транспонируя уравнение (1.90), найдем\par
\begin{center}
$E^{T}=\frac{1}{\rho}E$.     (1.81)\par
\end{center}
Таким образом, $\rho=1/\rho$, и, следовательно, $\rho=\pm1$. Значит, матрица \textsl{E}, определенная уравнением (1.69), или симметрична, или антисимметрична:\par
\begin{center}
$E^{T}=\pm E.$        (1.82)\par
\end{center}
Если взять матрицы $\gamma_1, \gamma_2, \dots ,\gamma_{\nu}$ симметричными,\par
а $\gamma_{\nu+1},\gamma_{\nu+2},\dots,\gamma_{2\nu}$ антисимметричными, то нетрудно видеть, что в качестве \textsl{E} можно взять матрицу\par
\begin{center}
$\textsl{E}=\lambda\gamma_1\gamma_2 \dots \gamma_{\nu}$,      (1.83)\par
\end{center}
если $\nu$ четно, и матрицу\par
\begin{center}
$\textsl{E}=\lambda\gamma_{\nu+1},\gamma_{\nu+2}, \dots, \gamma_{2\nu}$,     (1.84)\par
\end{center}
если $\nu$ нечетно. Здесь $\lambda\ne0$ - произвольное комплексное число.\par
Учитывая свойства симметрии матриц $\gamma_i$, взятых в выражениях (1.83), (1.84) для \textsl{E}, легко найти, что матрица \textsl{E} симметрична, если число 1/2$\nu(\nu+1)$ четное, и антисимметрична, если число 1/2$\nu(\nu+1)$ нечетное:\par
\begin{center}
$E^T=(-1)^{\frac{1}{2}\nu(\nu+1)}E$. (1.85)\par
\end{center}
Например, в случае четного $\nu$ для определения (1.83) имеем\par
$E^T=\gamma^{T}_{\nu}\dots \gamma^{T}_2\gamma^{T}_1=\gamma_{\nu}\dots \gamma_2\gamma_1=(-1)^{\frac{1}{2}\nu(\nu-1)}\gamma_1\gamma_2\dots \gamma_{\nu}=$\par
\begin{center}
$=(-1)^{\frac{1}{2}\nu(\nu-1)}E=(-1)^{\frac{1}{2}\nu(\nu+1)}E.$     (1.86) \par
\end{center}
Пользуясь уравнением (1.69), вычислим результат транспонирования произведения произвольных $\gamma$-матриц. Имеем $(\gamma_{i_1}\gamma_{i_2} \dots \gamma_{i_k})^{T}=$\par
\begin{center}
$=\gamma^{T}_{i_k} \dots \gamma^{T}_2\gamma^{T}_{i_1}=(-1)^{k}E\gamma_{i_k} \dots \gamma_{i_2}\gamma_{i_1}E^{-1}.$     (1.87)\par
\end{center}
Производя в этом равенстве альтернирование по индексам i и переставляя в правой его части индексы $i_k, \dots, i_2,i_1$ в порядке возрастания их номеров, получим \par
\begin{center}
$(\gamma_{i_1,i_2,\dots i_k})^{T}=(-1)^{\frac{1}{2}k(k+1)}E\gamma_{i_1,i_2,\dots i_k}E^{-1}$.     (1.88) \par
\end{center}
Учитывая свойства симметрии (1.85) матрицы \textsl{E} равенство (1.88) запишем в виде\par
\begin{center}
$(\textsl{E}\gamma_{i_1,i_2,\dots i_k})^{T}=(-1)^{\frac{1}{2}[\nu(\nu+1)+k(k+1)]}\textsl{E}\gamma_{i_1,i_2,\dots i_k}.$ (1.89)\par
\end{center}
Очевидно, что свойства симметрии (1.85), (1.89) носят инвариантный характер и не связаны с конкретным выбором матриц $\gamma_i$.\par
Обозначая элементы матриц $\textsl{E}\gamma_{i_1,i_2,\dots i_k}$ символом $\gamma_{BA i_1,i_2,\dots i_k}$.\par
\begin{center}
$\gamma_{BA i_1,i_2,\dots i_k}=e_{BC}\gamma^{C}_{A i_1 i_2 \dots i_k}$,      (1.90)\par
\end{center}
свойства симметрии (1.85), (1.89) можно записать также следующим образом: \par
\begin{center}
$e_{BA}=(-1)^{\frac{1}{2}\nu(\nu+1)}e_{AB}$,\par
$\gamma_{BA i_1 i_2 \dots i_k}=(-1)^{\frac{1}{2}[\nu(\nu+1)+k(k+1)]}\gamma_{AB i_1 i_2 \dots i_k}.$ (1.91)\par
\end{center}
\textbf
{\S. Спинорное представление ортогональной группы преобразований базисов четномерного комплексного евклидова пространства}\par
\textbf
{1. Спинорное представление собственной ортогональной группы.} Рассмотрим четномероное комплексное евклидово векторное пространство $E^{+}_{2\nu}$ размерности $2\nu$, отнесенное к ортонормированному базису \textsl{$Э_i$} (i=1, 2, \dots, $2\nu$). Пусть $SO^{+}_{2\nu}$-группа собственных ортогональных преобразований базисов $Э_i$ пространства $E^{+}_{2\nu}$:\par
\begin{center}
$Э_{i}^{'}=l^{j}_{i}Э_j$, (1.92)\par
\end{center}
определяемая уравнениями \par
\begin{center}
$l^{q}_{i}l^{m}_{j}\delta_{qm}=\delta_{ij}, det||l^{j}_{i}||=1$. (1.93)\par
\end{center}
Если матрицы $\gamma_i$ удовлетворяют уравнению (1.1), то из условия ортогональности (1.93) следует, что матрицы $\gamma^{'}_i$=\par
$=l^{j}_{i}\gamma_{j}$ также удовлетворяют уравнению (1.1). В самом \par
деле,\par
$\gamma^{'}_{i}\gamma^{'}_{j}+\gamma^{'}_{j}\gamma^{'}_{i}=l^{q}_{i}l^{m}_{j}(\gamma_q\gamma_m+\gamma_m\gamma_q)=2l^{q}_{i}l{m}_{j}\delta{qm}I=2\delta_{ij}I.$ (1.94)\par
Поэтому из теоремы Паули следует, что существует матрица\footnote[{1}]{S является квадратной матрицей порядка $2\nu$ с элементами $S^{B}_{A}$. Последующие соотношения в целях упрощения записи, как это обычно принято, будем в основном записывать в матричных обозначениях, опуская индексы, определяющие матричные элементы.}
S=$||S^{B}_{A}||$, определенная с точностью до умноже-
\end{document}
